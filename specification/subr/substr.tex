\clearpage
\phantomsection
\addcontentsline{toc}{subsection}{substr}
\label{subr:substr}
\subsection*{substr: return a sub string of a string}

\subsubsection*{Calling convention}

\begin{description}
\item[\registerop{rd}] a string representing the substring
\end{description}

\subsubsection*{Description}

The \subroutine{substr} routine returns a sub-string of a string,
passed as the first argument, starting from a byte index passed as the
second argument.  An optional third argument can be used to bound the
resulting string.  If the optional bounding argument is not supplied
then the sub-string includes all bytes up to and including the
terminating NUL character.

\subsubsection*{Pseudocode}

\begin{verbatim}
string = stack[0].value
index = stack[1].value
length = stack[1].value
%rd = substr(string, index, length)
\end{verbatim}

\subsubsection*{Constraints}

\subsubsection*{Failure modes}

This subroutine has no run-time failure modes beyond its constraints.
