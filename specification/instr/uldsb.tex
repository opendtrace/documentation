\clearpage
\phantomsection
\addcontentsline{toc}{subsection}{ULDSB}
\label{insn:uldsb}
\subsection*{ULDSB: load signed 8 bit quantity from user space}

\subsubsection*{Format}

\textrm{ULDSB \%rd, \%r1, \%r2}

\begin{center}
\begin{bytefield}[endianness=big,bitformatting=\scriptsize]{32}
\bitheader{0,7,8,15,16,23,24,31} \\
\bitbox{8}{0x40}
\bitbox{8}{r1}
\bitbox{8}{r2}
\bitbox{8}{rd}
\end{bytefield}
\end{center}

\subsubsection*{Description}

The \instruction{uldsb} instruction loads a signed 8 bit quantity from memory in
a user space process into the \registerop{rd} register, indexed by
\registerop{r1}. This instruction is a \textbf{signed} instruction and will
perform sign extension on the resulting register when applicable.

\subsubsection*{Pseudocode}

\begin{verbatim}
%rd = umem[r1]
\end{verbatim}

\subsubsection*{Load-time constraints}
The registers \registerop{r1} and \registerop{rd} must be valid registers,
\registerop{r2} must be \registerop{r0} and \registerop{rd} must not be
\registerop{r0}.

\subsubsection*{Failure modes}

This instruction can cause a page fault if the memory it is trying to access
is not paged in.
