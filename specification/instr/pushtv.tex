\clearpage
\phantomsection
\addcontentsline{toc}{subsection}{PUSHTV}
\label{insn:pushtv}
\subsection*{PUSHTV: push a value onto the stack}

\subsubsection*{Format}

\textrm{PUSHTV \%rs}

\begin{center}
\begin{bytefield}[endianness=big,bitformatting=\scriptsize]{32}
\bitheader{0,7,8,15,16,23,24,31} \\
\bitbox{8}{0x31}
\bitbox{8}{r0}
\bitbox{8}{r0}
\bitbox{8}{rs}
\end{bytefield}
\end{center}

\subsubsection*{Description}

The \instruction{pushtv} instruction takes the value contained in
\registerop{rs} register and pushes it onto the stack.  Unlike the
\verb|PUSHTR| instruction, the size of the value is \emph{not} stored
along with the value.

\subsubsection*{Pseudocode}

\begin{verbatim}
stack[++index].value = %rs
stack[index].size = 0;
\end{verbatim}

\subsubsection*{Failure modes}

This instruction has no run-time failure modes beyond its constraints.
