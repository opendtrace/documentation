\clearpage
\phantomsection
\addcontentsline{toc}{subsection}{BGU}
\label{insn:bgu}
\subsection*{BGU: branch greater than, unsigned}

\subsubsection*{Format}

\textrm{BGU label}

\begin{center}
\begin{bytefield}[endianness=big,bitformatting=\scriptsize]{32}
\bitheader{0,23,24,31} \\
\bitbox{8}{0x15}
\bitbox{24}{label}
\end{bytefield}
\end{center}

\subsubsection*{Description}

The \instruction{bgu} instruction sets the \registerop{pc} to the new
label if, and only if, the result of the previous comparison shows
that \registerop{r1} was greater than \registerop{r2}.

\subsubsection*{Pseudocode}

\begin{verbatim}
if ((cc_c | cc_z) == 0)
	pc = label;
\end{verbatim}

\subsubsection*{Load-time constraints}
DIF does not allow backwards branches, therefore \verb+label+ must be
greater than \verb+pc+. Moreover, \verb+label+ must not go past the last
address of the current DIF object.

\subsubsection*{Failure modes}

This instruction has no run-time failure modes beyond its constraints.
