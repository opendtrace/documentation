\clearpage
\phantomsection
\addcontentsline{toc}{subsection}{SRA}
\label{insn:sra}
\subsection*{SRA: Shift Right Arithmetic}

\subsubsection*{Format}

\textrm{SRA \%rd, \%r1, \%r2}

\begin{center}
\begin{bytefield}[endianness=big,bitformatting=\scriptsize]{32}
\bitheader{0,7,8,15,16,23,24,31} \\
\bitbox{8}{0x2E}
\bitbox{8}{r1}
\bitbox{8}{r2}
\bitbox{8}{rd}
\end{bytefield}
\end{center}

\subsubsection*{Description}

The \instruction{sra} instruction shifts the value in \registerop{r1} right by
the number of bits indicated in \registerop{r2}, placing the results in register
\registerop{rd}.  This instruction only operates on \textbf{signed} integers.

\subsubsection*{Pseudocode}

\begin{verbatim}
%rd = %r1 >> %r2
\end{verbatim}

\subsubsection*{Constraints}

\registerop{r1}, \registerop{r2}, and \registerop{rd} must be less than
\nregs{}.

\medskip
\noindent
\registerop{rd} cannot be \register{r0}.

\subsubsection*{Failure modes}

This instruction has no run-time failure modes beyond its constraints.
