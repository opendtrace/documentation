\clearpage
\phantomsection
\addcontentsline{toc}{subsection}{XLATE}
\label{insn:xlate}
\subsection*{XLATE: }

\subsubsection*{Format}

\textrm{XLATE \%rd, \%r1, \%r2}

\begin{center}
\begin{bytefield}[endianness=big,bitformatting=\scriptsize]{32}
\bitheader{0,7,8,15,16,23,24,31} \\
\bitbox{8}{0x4E}
\bitbox{8}{r1}
\bitbox{8}{r2}
\bitbox{8}{rd}
\end{bytefield}
\end{center}

\subsubsection*{Description}

\emph{NOTE:} This instruction is not used by the kernel as all
translations are handled in user space.

The \instruction{xlate} instruction extracts translated data indicated
at current the current translation index and returns the data in
\registerop{rd}.

\subsubsection*{Pseudocode}

\subsubsection*{Constraints}

\subsubsection*{Failure modes}

This instruction has no run-time failure modes beyond its constraints.
