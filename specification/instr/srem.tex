\clearpage
\phantomsection
\addcontentsline{toc}{subsection}{SREM}
\label{insn:srem}
\subsection*{SREM: divide two numbers and store the remainder}

\subsubsection*{Format}

\textrm{SREM \%rd, \%r1, \%r2}

\begin{center}
\begin{bytefield}[endianness=big,bitformatting=\scriptsize]{32}
\bitheader{0,7,8,15,16,23,24,31} \\
\bitbox{8}{0x0B}
\bitbox{8}{r1}
\bitbox{8}{r2}
\bitbox{8}{rd}
\end{bytefield}
\end{center}

\subsubsection*{Description}

The \instruction{srem} instruction divides the value contained in
\registerop{r2} into that contained in \registerop{r1} placing the remainder
into \registerop{rd}.  The values in both \registerop{r1} and \registerop{r2}
are first promoted to signed, 64 bit values, before the division operation is
carried out. The \instruction{srem} instruction follows the remainder definition
in C99 and will return a negative remainder if applicable.

\subsubsection*{Pseudocode}

\begin{verbatim}
%rd = (int64_t)%r1 % (inst64_t)%r2
\end{verbatim}

\subsubsection*{Constraints}

\subsubsection*{Failure modes}

This instruction has no run-time failure modes beyond its constraints.
