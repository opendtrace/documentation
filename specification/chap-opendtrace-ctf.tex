The Compact C Type Format (CTF) encapsulates all of the information
needed by OpenDTrace to understand C language types such as integers,
strings, floats and structures.  The goal of having another section
just for C type information is to provide a compact representation of
the information that usually appears in the debugging sections of
object files and executables.  CTF only contains data types it
does not contain other debugging infromation, which allows it to be
far more compact.  The debugging sections on a debug build of FreeBSD
in 2017 take up 78 megabytes of space, while the CTF section in the
same kernel take up only 800 kilobytes. 

\section{On Disk Format}
\label{sec:ctf-on-disk-format}

CTF data is stored in its own ELF section within an object file or
executable.  It is meant to be stored in a format that is both compact
and which is properly aligned so that it can be accessed using the
mmap(2) system call.

\begin{figure}[h]
  \centering
  \includegraphics[width=.8\textwidth]{ctf-stable-format}
  \caption{CTF Stable Storage Format}
  \label{fig:ctf-stable-storage-format}
\end{figure}

Figure~\ref{fig:ctf-stable-storage-format} shows all of the components
of the CTF section as they would be found on stable storage.  The file
header stores a magic number and version information, encoding flags,
and the byte offset of each of the sections relative to the end of the
header itself.  As of this writing the most current version of CTF is
version two (2).  The preamblem including the magic number, version
and flags take up the first 32 bits of the header, the remaining
fields take up 32 bits each, independent of the word size of the
architecture.

\begin{center}
\begin{bytefield}[endianness=big,bitformatting=\scriptsize]{32}
\bitheader{0,8,16,31} \\
\bitbox{16}{magic}
\bitbox{8}{version}
\bitbox{8}{flags}\\
\bitbox{32}{reference to parent label}\\
\bitbox{32}{reference to basename of parent}\\
\bitbox{32}{label section offset}\\
\bitbox{32}{function section offset}\\
\bitbox{32}{type section offset}\\
\bitbox{32}{string section offset}\\
\bitbox{32}{size of string section (bytes)}\\
\end{bytefield}
\end{center}

The CTF section makes heavy use of references between the sub-sections
to fully describe the datatypes in a program as well as the functions,
the function's argument list, and the function's return value.  The
\verb|data objects| and \verb|functions| sections depend upon the
\verb|types| section, which encodes all of the datatypes that have
been during the CTF conversion process.  Each type has a unique number
and name, as well as a size and encoding.  Types may refer to other,
more primitive types by uses of a reference, e.g. a \verb|uint32_t|
will actually refer to a \verb|unsigned int|.  Types are broken up by
what they represent, referred to as their \emph{kind}.


\begin{tabular}{|l|l}
\hline
\verb|CTF_K_UNKNOWN| & unknown type (used for padding)\\
\verb|CTF_K_INTEGER| & variant data is \verb|CTF_INT_DATA()| (see below)\\
\verb|CTF_K_FLOAT| & variant data is \verb|CTF_FP_DATA()| (see below)\\
\verb|CTF_K_POINTER| & \verb|ctt_type| is referenced type\\
\verb|CTF_K_ARRAY| & variant data is single \verb|ctf_array_t|\\
  \verb|CTF_K_FUNCTION| & \verb|ctt_type| is return type\\
  variant data is list of argument types (\verb|ushort_t|'s)\\
\verb|CTF_K_STRUCT| & variant data is list of \verb|ctf_member_t|'s\\
\verb|CTF_K_UNION| & variant data is list of \verb|ctf_member_t|'s\\
\verb|CTF_K_ENUM| & variant data is list of \verb|ctf_enum_t|'s\\
\verb|CTF_K_FORWARD| & no additional data; \verb|ctt_name| is tag\\
\verb|CTF_K_TYPEDEF| & \verb|ctt_type| is referenced type\\
\verb|CTF_K_VOLATILE| & \verb|ctt_type| is base type\\
\verb|CTF_K_CONST| & \verb|ctt_type| is base type\\
\verb|CTF_K_RESTRICT| & \verb|ctt_type| is base type\\
\hline
\end{tabular}

Complex data types, such as structures, are also contained in the
\verb|types| section, and are encoded as a structure with a name that
references the string table.

\begin{center}
\begin{bytefield}[endianness=big,bitformatting=\scriptsize]{32}
\bitheader{0,16,31} \\
\bitbox{32}{name}\\
\bitbox{16}{info}
\bitbox{16}{size or type}\\
\end{bytefield}
\end{center}

A simple type, one who's size is less than 64 Kbytes, are stored
in a \verb|ctf_stype|.  The \verb|name| is a reference to a string
in the string table.  The \verb|info| field is encoded differently
for each type, as will be explained fully in the rest of this chapter.
The last field is either the size, in bytes, of the structure or it
is a reference to another type, encoded using the referenced type's
ID.  The majority of types in a C program will fit within a
\verb|ctf_stype|.

\begin{center}
\begin{bytefield}[endianness=big,bitformatting=\scriptsize]{32}
\bitheader{0,16,31} \\
\bitbox{32}{name}\\
\bitbox{16}{info}
\bitbox{16}{size or type}\\
\bitbox{32}{high 32 bits of size (in bytes)}\\
\bitbox{32}{low 32 bits of size (in bytes)}
\end{bytefield}
\end{center}

Types that are larger than 64Kbytes are encoded using a \verb|ctf_type|
structure.  The \verb|name| and \verb|info| fields of this, larger,
\verb|ctf_type| are the same as the smaller \verb|ctf_stype|, but the
\verb|size| field is always set to \verb|CTF_LSIZE_SENT|, the sentinal
value that tells the consumer that this is a larger structure.  A
\verb|ctf_type| structure can encode an extremely large type, since
it provides 64 bits for the size, and that size is being expressed in
bytes.

\begin{center}
\begin{bytefield}[endianness=big,bitformatting=\scriptsize]{16}
\bitheader{0,9,10,15} \\
\bitbox{5}{kind}\\
\bitbox{1}{isroot}\\
\bitbox{10}{vlen}
\end{bytefield}
\end{center}

The \verb|info| field is further broken down into a number of sub-fields
which encoded the \verb|kind|, \verb|vlen| (variable length) and whether
or not this is a root type \verb|isroot|.

Each of the integral types, such as integers, floats, pointers, arrays, etc.
has its own encoding.  Integers are the simplest type and are unsigned by
default.  An integer type is encoded in a single, 32 bit, field

\begin{center}
\begin{bytefield}[endianness=big,bitformatting=\scriptsize]{32}
\bitheader{0,16,24,31} \\
\bitbox{8}{flags}\\
\bitbox{8}{offset}\\
\bitbox{16}{size in bits}
\end{bytefield}
\end{center}

The \verb|flags| field indicates whether the integer is signed,
contains character data, is a boolean or is to be displayed
with a varags style of formatting.

Floating point numbers have the exact same fields to describe them
but a larger number of possible flags, to match the larger
number of ways in which floating point numbers may be stored.
The flags and descriptions of the currently supported floating
point encodings are given in \ref{}

begin{tabular}{|l|l}
\hline
CTF_FP_SINGLE	& IEEE 32-bit float encoding\\
CTF_FP_DOUBLE	& IEEE 64-bit float encoding\
CTF_FP_CPLX	& Complex encoding\\
CTF_FP_DCPLX	& Double complex encoding\\
CTF_FP_LDCPLX	& Long double complex encoding\\
CTF_FP_LDOUBLE	& Long double encoding\\
CTF_FP_INTRVL	& Interval (2x32-bit) encoding\\
CTF_FP_DINTRVL	& Double interval (2x64-bit) encoding\\
CTF_FP_LDINTRVL	& Long double interval (2x128-bit) encoding\\
CTF_FP_IMAGRY	& Imaginary (32-bit) encoding\\
CTF_FP_DIMAGRY	& Long imaginary (64-bit) encoding\\
CTF_FP_LDIMAGRY	& Long long imaginary (128-bit) encoding
\hline
\end{tabular}

The functions section encodes the function name, as well as its arguments
and return value.  The types of the arguments and the return value
reference the \verb|types| section.  The arguments to the function
are encoded as a list.

All strings are encoded in the \verb|string table| and are referenced by
a numeric id from the other sections.
l
%%% Local Variables:
%%% mode: latex
%%% TeX-master: "dtrace-specification"
%%% End:
