OpenDTrace specifies a Application Binary Interface (ABI) between kernel and
userspace in the form of trace and aggregations buffers and the associated
metadata used to interpret these buffers (for example, for further processing
or formatting in order to generate results pased back to the user). This
chapter specifies the format of the OpenDTrace trace and aggregation buffers,
and metadata data structures used to interpret them.

\section{Enabling}
\label{sec:probes}

Each enabling (that is, an enabled probe) is associated with a set of
\textit{actions} through its \textit{Enabling Control Block} (ECB). When a
probe fires (and, if present, its predicate evaluates to true) these
actions are performed. The execution of these actions potential results in
data being written into the trace buffers.

\subsection{OpenDTrace trace buffer}
\label{subsec:trace_buffer}

Each OpenDTrace consumer is associated with a set of in-kernel, per-CPU buffers
\cite{DTrace2004}.  The format of the OpenDTrace trace buffer is shown in
Figure~\ref{fig:trace_buffer}.  Note that the length of the data for the trace
record is not specified by the OpenDTrace trace buffer (trace records are specified
as TV (\textit{Type-Value}) rather than TLV (\textit{Type-Length-Value}).
Instead, a separate stream of metadata is used to interpret the trace buffer.
The data structures describing the metadata stream are described in
Section~\ref{subsec:probe_data_structures}.

\begin{figure}[!ht]
	\centering
    \begin{tikzpicture}
		\coordinate[] (origin) at (0,0);
		\node[
			rectangle split,
			rectangle split horizontal=true,
			minimum height=1cm,
			align=center,
			rectangle split parts=7,
			rectangle split part fill={red!20,blue!20,green!20,none,red!20,blue!20,green!20},
			draw
		] (barr1) [above=5pt of origin]
		{	\nodepart[nn]{one}\texttt{EPID}%
			\nodepart[nn]{two}\texttt{timestamp}%
			\nodepart[nn, text width=3cm]{three}\texttt{data}%
			\nodepart[nn]{four}...%
			\nodepart[nn]{five}\texttt{EPID}%
			\nodepart[nn]{six}\texttt{timestamp}%
			\nodepart[nn,text width=3cm]{seven}\texttt{data}%
		};
	\draw [db] (barr1.south west) -- (barr1.three split south) node[midway,below=10pt]{$Record_0$};
	\draw [db] (barr1.four split south) -- node[below=10pt]{$Record_n$} (barr1.south east);
    \end{tikzpicture}
	\caption{OpenDTrace trace buffer format.}
	\label{fig:trace_buffer}
\end{figure}

\begin{itemize}
	\item{\texttt{EPID}}: identifies the enabling (that is, the enabled
		probe) that produced the trace record; the identifier type is
		\texttt{dtrace\_epid\_t} which corresponds to a unsigned 32 bit
		integer.  These identifiers are unqiue for each OpenDTrace
		consumer.

	\item{\texttt{timestamp}:} the timestamp (ns) of the trace record; the
		timestamp type is an unsigned 64 bit integer.

	\item{\texttt{data}:} the data for trace record; a sequence of octets
		the format of which is specified by the trace metadata see
		Section~\ref{subsec:probe_data_structures}.
\end{itemize}

\subsection{Trace metadata}
\label{subsec:probe_data_structures}

The metadata required to interpret an enabled probe is constant over
the lifetime of the enabling \cite{DTrace2004}. This allows trace metadata
to be queried from the kernel once (on first processing a given
enabling) and then locally cached.  The separation of trace records and
the metadata required to interpret records is a central design decision.
This seperation simplifies the run-time analysis of the trace data but comes
at the expense of some flexibility (for example, the ability change an
enabling at runtime).

\tikzset{
    %every node/.style={draw, text height=1.5ex},
    split/.style={rectangle split, rectangle split parts=#1,draw,
        rectangle split horizontal=false,rectangle split part align=base},
}

Figure~\ref{fig:overview} provides an overview of the data structures (and
their relationships) used by \texttt{libdtrace} when interpreting the contents
of a OpenDTrace trace buffer.

\begin{figure}[!htp]
	\centering
	\begin{tikzpicture}[->, -{Latex}]

		\node[split=9,label=north:\texttt{struct~\textit{dtrace\_probedata}}] (a1) at (0,-9) {
		\nodepart{one}\textit{dtpda\_handle: dtrace\_hdl\_t~*}
		\nodepart{two}\textit{pdtpda\_edesc}
		\nodepart{three}\textit{dtpda\_pdesc}
		\nodepart{four}\textit{dtpda\_cpu: processorid\_t}
		\nodepart{five}\textit{dtpda\_data: caddr\_t}
		\nodepart{six}\textit{dtpda\_flow: dtraace\_flowkind\_t}
		\nodepart{seven}\textit{dtpda\_prefix: const~char~*}
		\nodepart{eight}\textit{dtpda\_indent: int}
		\nodepart{nine}\textit{dtpda\_timestamp: uint64\_t}};

		\node[split=6,label=north:\texttt{struct~\textit{dtrace\_eprobedesc}}] (a2) at (6,-7) {
		\nodepart{one}\textit{dtepd\_epid: dtrace\_epid\_t}
		\nodepart{two}\textit{dtepd\_probeid: dtrace\_id\_t}
		\nodepart{three}\textit{dtepd\_uarg: uint64\_t}
		\nodepart{four}\textit{dtepd\_size: uint32\_t}
		\nodepart{five}\textit{dtepd\_nrec: int}
		\nodepart{six}\textit{dtepd\_rec}};

		\node[split=5,label=south:\texttt{struct~\textit{dtrace\_probedesc}}] (a3) at (6,-12) {
		\nodepart{one}\textit{dtpd\_id: dtrace\_id\_t}
		\nodepart{two}\textit{dtpd\_provider: char []}
		\nodepart{three}\textit{dtpd\_mod: char []}
		\nodepart{four}\textit{dtpd\_func: char []}
		\nodepart{five}\textit{dtpd\_name: char []}};

		\node[split=6,label=north:\texttt{struct~\textit{dtrace\_recdesc\_t}}] (recdesc) at (12,-6) {
		\nodepart{one}\textit{dtrd\_action: dtrace\_actkind\_t}
		\nodepart{two}\textit{dtrd\_size: uint32\_t}
		\nodepart{three}\textit{dtrd\_offset: uint32\_t}
		\nodepart{four}\textit{dtrd\_alignment: uint16\_t}
		\nodepart{five}\textit{dtrd\_format: uint16\_t}
		\nodepart{six}\textit{dtrd\_arg: uint64\_t}
		\nodepart{seven}\textit{dtrd\_uarg: uint64\_t}};

		\draw (a1.two east) -- (a2.one west);
		\draw (a1.three east) -- (a3.one west);
		\draw (a2.six east) -- (recdesc.one west);

		\node[split=5,label=north:\texttt{struct~\textit{dt\_pfarvg}}] (pfargv) at (0,0) {
		\textit{pfv\_dtp: dtrace\_hdl\_t *}
		\nodepart{two}\textit{pfv\_format: char *}
		\nodepart{three}\textit{pfv\_argv}
		\nodepart{four}\textit{pfv\_argc: uint\_t}
		\nodepart{five}\textit{pfv\_flags: uint\_t}};

		\node[split=10,label=north:\texttt{struct~\textit{dt\_pfarvgd}}] (pfargd) at (6,0) {
		\textit{pfd\_prefix: const~char~*}
		\nodepart{two}\textit{pfd\_preflen: char~*}
		\nodepart{three}\textit{pfd\_fmt: char[8]}
		\nodepart{four}\textit{pfd\_flags: uint\_t}
		\nodepart{five}\textit{pfd\_width: int}
		\nodepart{six}\textit{pfd\_dynwidth: int}
		\nodepart{seven}\textit{pfd\_prec: int}
		\nodepart{eight}\textit{pfd\_conv}
		\nodepart{nine}\textit{pfd\_rec}
		\nodepart{ten}\textit{pfd\_next: struct dt\_pfargd~*}};

		\node[split=10,label=north:\texttt{struct~\textit{dt\_pfconv}}] (pfconv) at (12,0) {
		\textit{pfc\_name : const~char~*}
		\nodepart{two}\textit{pfc\_ofmt: const~char~*}
		\nodepart{three}\textit{pfc\_tstr: const~char~*}
		\nodepart{four}\textit{pfc\_check: dt\_pfcheck\_f}
		\nodepart{five}\textit{pfc\_print: dt\_pfprint\_f}
		\nodepart{six}\textit{pfc\_cctfp: ctf\_file\_t~*}
		\nodepart{seven}\textit{pfc\_ctype: ctf\_id\_t}
		\nodepart{eight}\textit{pfc\_dctfp: ctf\_file\_t~*}
		\nodepart{nine}\textit{pfc\_dtype: ctf\_id\_t}
		\nodepart{ten}\textit{pfc\_next: struct dt\_conv~*}};

		\draw (pfargv.three east) -- (pfargd.one west);
		\draw (pfargd.eight east) -- (pfconv.one west);
		\draw (pfargd.nine east) -- (recdesc.one west);

	\end{tikzpicture}
	\caption{Overview of the data structures used to interpret the OpenDTrace
	trace buffer}
	\label{fig:overview}
\end{figure}

\subsubsection{\texttt{struct \textit{dtrace\_probedata}}}
\label{subsubsec:dtpda}

The \texttt{struct~\textit{dtrace\_probedata}} (Figure~\ref{fig:dtpda}) is used
solely by \texttt{libdtrace}. This data structure aggregates the information
required to process the OpenDTrace trace buffer including metadata describing the
enabling, a pointer to the copy of the trace buffer and formatting information
(such as the flow prefix and indent---used when the OpenDTrace is invoked with the
\texttt{flowindent} option enabled).

\begin{figure}[!ht]
	\centering
    \begin{tikzpicture}[->, -{Latex}]
		\node[split=9,label=north:\texttt{struct \textit{dtrace\_probedata}}] (a1) at (0,-9) {
		\textit{dtpda\_handle: dtrace\_hdl\_t~*}
		\nodepart{two}\textit{pdtpda\_edesc}
		\nodepart{three}\textit{dtpda\_pdesc}
		\nodepart{four}\textit{dtpda\_cpu: processorid\_t}
		\nodepart{five}\textit{dtpda\_data: caddr\_t}
		\nodepart{six}\textit{dtpda\_flow: dtraace\_flowkind\_t}
		\nodepart{seven}\textit{dtpda\_prefix: const~char~*}
		\nodepart{eight}\textit{dtpda\_indent: int}
		\nodepart{nine}\textit{dtpda\_timestamp: uint64\_t}};
    \end{tikzpicture}
	\caption{Data structure used to aggregate the details of the trace buffer and the metadata required to interpret it.}
	\label{fig:dtpda}
\end{figure}

\begin{itemize}

	\item The \texttt{dtpda\_handle} field contains a pointer to the handle
returned to OpenDTrace consumer on invoking \texttt{dtrace\_open()}.

	\item The \texttt{dtpda\_edesc} and \texttt{dtpda\_pdesc} fields are
described in Sections~\ref{subsubsec:dtepd} and~\ref{subsubsec:dtpd}
respectively.

	\item The \texttt{dtpda\_cpu} field identifies the CPU on which the
probe fired.

	\item The \texttt{dtpda\_data} field contains a pointer to the OpenDTrace
trace buffer (see Section~\ref{subsec:trace_buffer}).

	\item The \texttt{dtpda\_flow} field specifies the flow type (either
DTRACEFLOW\_ENTRY, DTRACEFLOW\_RETURN or DTRACEFLOW\_NONE). The \texttt{flow}
field is set when the DTrace option \texttt{flowindent} is true; the value of
\texttt{dtpda\_flow} depends on whether a return (\texttt{::return}) or entry
(\texttt{::entry}) probe is being traced.

	\item The \texttt{dtpda\_prefix} field contains a pointer to a C String
containing the flow prefix (nominally ``-$\rangle$'' for entry probes and
``$\langle$-'' for return probes).

	\item The \texttt{dtpda\_indent} field specifies the value of the flow
indent (that is the number of characters currently indented).

	\item The \texttt{dtpda\_timestamp} field contains the timestamp of the
trace record extracted from the OpenDTrace trace buffer (see
Section~\ref{subsec:trace_buffer}).

\end{itemize}

\subsubsection{\texttt{struct \textit{dtrace\_eprobedesc}}}
\label{subsubsec:dtepd}

The \texttt{struct~\textit{dtrace\_eprobedesc}} (Figure~\ref{fig:dtepd})
specifies an enabled probe. Specifically, \texttt{struct~\textit{eprobedesc}}
contains metadata required to process the OpenDTrace trace buffer. The
\texttt{struct~\textit{dtrace\_eprobedesc}} is returned from the kernel to
userspace through invoking the \texttt{EPROBE} ioctl.

\begin{figure}[!ht]
	\centering
	\begin{tikzpicture}[->, -{Latex}]

		\node[split=6,label=north:\texttt{struct~\textit{dtrace\_eprobedesc}}] (a2) at (6,-7) {
		\nodepart{one}\textit{dtepd\_epid: dtrace\_epid\_t}
		\nodepart{two}\textit{dtepd\_probeid: dtrace\_id\_t}
		\nodepart{three}\textit{dtepd\_uarg: uint64\_t}
		\nodepart{four}\textit{dtepd\_size: uint32\_t}
		\nodepart{five}\textit{dtepd\_nrec: int}
		\nodepart{six}\textit{dtepd\_rec}};
	\end{tikzpicture}
	\caption{Data structure used to describe an enabled probe.}
	\label{fig:dtepd}
\end{figure}

\begin{itemize}

	\item The \texttt{dtepd\_epid} field contains the enabled probe's
identifier; the identifier type is \texttt{dtrace\_epid\_t} which corresponds 
to a unsigned 32 bit integer.

	\item The \texttt{dtpd\_id} field contains the probe's identifier; the
identifier type is \texttt{dtrace\_id\_t} which corresponds to a unsigned 32 bit
integer.

	\item The \texttt{dtepd\_uarg} field is a library argument\footnote{I'm
uncertain when this is used, and whether it is relevant when in printing
trace records}. 

	\item The \texttt{dtepd\_size} field specifies the size of the OpenDTrace
trace buffer (a pointer to the trace buffer is held in the
\texttt{struct~\textit{dtrace\_probedata}} structure.

	\item The \texttt{dtepd\_nrec} field specifies the number of records in
the \texttt{dtepd\_rec} field.

	\item The \texttt{dtepd\_rec} field is a variable sized array (of
\texttt{dtepd\_nrecs} entries); this data structure is described in
Section~\ref{subsubsec:dtrd}.

\end{itemize}

\subsubsection{\texttt{struct \textit{dtrace\_probedesc}}}
\label{subsubsec:dtpd}

The \texttt{struct~\textit{dtrace\_probedesc}} (Figure~\ref{fig:dtpd})
specifies a given probe. The \texttt{\textit{dtrace\_probedesc}} structure is
constructed within the kernel from the corresponding
\texttt{struct~\textit{dof\_probedesc}}. The
\texttt{struct~\textit{dtrace\_probedesc}} is returned from the kernel to userspace
by invoking the \texttt{PROBES} ioctl.

\begin{figure}[!ht]
	\centering
	\begin{tikzpicture}[->, -{Latex}]

		\node[split=5,label=north:\texttt{struct~\textit{dtrace\_probedesc}}] (a3) at (6,-12) {
		\textit{dtpd\_id: dtrace\_id\_t}
		\nodepart{two}\textit{dtpd\_provider: char[]}
		\nodepart{three}\textit{dtpd\_mod: char[]}
		\nodepart{four}\textit{dtpd\_func: char[]}
		\nodepart{five}\textit{dtpd\_name: char[]}};
	\end{tikzpicture}
	\caption{Data structure used to describe a probe.}
	\label{fig:dtpd}
\end{figure}

\begin{itemize}

	\item The \texttt{dtpd\_id} field contains the probe's identifier; the
identifier type is \texttt{dtrace\_id\_t} which corresponds to a unsigned 32 bit
integer.

	\item The \texttt{dtpda\_provider} field contains a C string specifying
the probe's provider name; the provider type is an \texttt{char} array or size
\texttt{DTRACE\_PROVNAMELEN} (nominally 64 characters).

	\item The \texttt{dtpda\_mod} field contains a C string specifying
the probe's module name; the provider type is an \texttt{char} array or size
\texttt{DTRACE\_MODNAMELEN} (nominally 64 characters).

	\item The \texttt{dtpda\_func} field contains a C string specifying
the probe's function name; the provider type is an \texttt{char} array or size
\texttt{DTRACE\_FUNCNAMELEN} (nominally 192 characters).

	\item The \texttt{dtpda\_name} field contains a C string specifying
the probe's name; the provider type is an \texttt{char} array or size
\texttt{DTRACE\_NAMELEN} (nominally 64 characters).

\end{itemize}

\subsubsection{\texttt{struct~\textit{dtrace\_recdesc}}}
\label{subsubsec:dtrd}

The \texttt{struct~\textit{dtrace\_recdesc}} (Figure~\ref{fig:dtrd}) specifies
an individual trace record within the trace buffer. Each OpenDTrace action produces
a separate trace record. And actions have a \textit{one-to-one} correspondence
with a DIFO (DTrace Intermediate Format Object). For example,
\texttt{printf("\%s", probefunc)} produce a single DIFO and therefore a single
action and record. Whereas, \texttt{prinf("\%s \%s", probefunc, arg0);} will
produce two DIFOs and therefore two actions and records.

\begin{figure}[!ht]
	\centering
	\begin{tikzpicture}[->, -{Latex}]

		\node[split=7,label=north:\texttt{struct~\textit{dtrace\_recdesc\_t}}] (recdesc) at (12,-6) {
		\nodepart{one}\textit{dtrd\_action: dtrace\_actkind\_t}
		\nodepart{two}\textit{dtrd\_size: uint32\_t}
		\nodepart{three}\textit{dtrd\_offset: uint32\_t}
		\nodepart{four}\textit{dtrd\_alignment: uint16\_t}
		\nodepart{five}\textit{dtrd\_format: uint16\_t}
		\nodepart{six}\textit{dtrd\_arg: uint64\_t}
		\nodepart{seven}\textit{dtrd\_uarg: uint64\_t}};

	\end{tikzpicture}
	\caption{Data structure used to describe a individual record in the trace buffer.}
	\label{fig:dtrd}
\end{figure}

\begin{itemize}

	\item The \texttt{dtrd\_action} field specifies the ``action'' of the
trace record; for \texttt{printf} the action is DTRACEACT\_PRINTF. The value of
\texttt{dtrd\_action} is used by \texttt{libdtrace} to determine how the trace
record is processed.

	\item The \texttt{dtrd\_size} field contains the size (in bytes) of the
trace record; this is computed within the kernel based on the CTF (Compact Type
Formet) type of the value being written to the trace buffer.

	\item The \texttt{dtrd\_offset} field contains the offset (in bytes)
into the OpenDTrace trace buffer at which the trace record is located; the offset
is computed within the kernel and is based on the size and alignment of the
preceeeding records.

	\item The \texttt{dtrd\_alignment} field contains the specified
alignment (in bytes) of the trace record; the alignment is based on the trace
record's CTF type.

	\item The \texttt{dtrd\_format} field contains an identifier used to lookup
format information by invoking the function \texttt{dt\_format\_lookup()}.
Format data is copied from the kernel to the userspace consumer by invoking the
\texttt{FORMAT} ioctl. In the case of a \texttt{printf} action the format data
is stored as a \texttt{struct~\textit{dt\_pfargv} (see
Section~\ref{subsubsec:pfv})}. 

	\item The \texttt{dtrd\_arg} field is unused when printing.

	\item The \texttt{dtrd\_uarg} field is unused when printing.

\end{itemize}

\subsubsection{\texttt{struct~\textit{dt\_pfargv}}}
\label{subsubsec:pfv}

The \texttt{struct~\textit{dt\_pfargv}} (Figure~\ref{fig:pfv}) is used solely
by \texttt{libdtrace}. This data structure acts as a container for data
structures that define a set of format codes (such as \texttt{\%s} or
\texttt{\%2d}) used by the \texttt{printf} action.

\begin{figure}[!ht]
	\centering
	\begin{tikzpicture}[->, -{Latex}]

		\node[split=5,label=north:\texttt{struct~\textit{dt\_pfarvg}}] (pfargv) at (0,0) {
		\textit{pfv\_dtp: dtrace\_hdl\_t~*}
		\nodepart{two}\textit{pfv\_format: char~*}
		\nodepart{three}\textit{pfv\_argv}
		\nodepart{four}\textit{pfv\_argc: uint\_t}
		\nodepart{five}\textit{pfv\_flags: uint\_t}};

	\end{tikzpicture}
	\caption{Data structure used to describe the formating of a \texttt{printf} action.}
	\label{fig:pfv}
\end{figure}

\begin{itemize}

	\item The \texttt{pfv\_handle} field contains a pointer to the handle
returned to OpenDTrace consumer on invoking \texttt{dtrace\_open()}.

	\item The \texttt{pfv\_format} field points to a C String containing the
format string. For example, if the action is
\texttt{printf("$\backslash$"event$\backslash$": \%s", probefunc);}, the format
string contains \texttt{"$\backslash$"event$\backslash$": \%s"}. 

	\item The \texttt{pfv\_argv} field is described in
Section~\ref{subsubsec:pfd}.

	\item The \texttt{pfv\_argc} field contains the number of entries in the
list pointed to by the \texttt{pfv\_argv} field.

	\item The \texttt{pfv\_flags} field contains flags used for validating
the the format arguments.

\end{itemize}

\subsubsection{\texttt{struct~\textit{dt\_pfargd}}}
\label{subsubsec:pfd}

The \texttt{struct~\textit{dt\_pfargd}} (Figure~\ref{fig:pfd}) is used
solely by \texttt{libdtrace}. This data structure defines an individual
\texttt{printf} format code (such as \texttt{\%s}).

\begin{figure}[!h]
	\centering
	\begin{tikzpicture}[->, -{Latex}]

		\node[split=10,label=north:\texttt{struct~\textit{dt\_pfarvgd}}] (pfargd) at (6,0) {
		\textit{pfd\_prefix: const~char~*}
		\nodepart{two}\textit{pfd\_preflen: char~*}
		\nodepart{three}\textit{pfd\_fmt: char[8]}
		\nodepart{four}\textit{pfd\_flags: uint\_t}
		\nodepart{five}\textit{pfd\_width: int}
		\nodepart{six}\textit{pfd\_dynwidth: int}
		\nodepart{seven}\textit{pfd\_prec: int}
		\nodepart{eight}\textit{pfd\_conv}
		\nodepart{nine}\textit{pfd\_rec}
		\nodepart{ten}\textit{pfd\_next: struct dt\_pfargd~*}};

	\end{tikzpicture}
	\caption{Data structure used to describe a format code used by the \texttt{printf} action.}
	\label{fig:pfd}
\end{figure}

\begin{itemize}

	\item The \texttt{pfd\_prefix} field points to C string that contains
the format string passed to \texttt{printf{}}.

	\item The \texttt{pfd\_preflen} field contains the length in bytes of the
prefix (\texttt{pfd\_prefix}). For example, for the prefix
"$\backslash$"event$\backslash$": \%s" the \texttt{pfd\_preflen} will be 9 (the
number of characters preceeding the format code \texttt{\%s}). 

	\item The \texttt{pfd\_fmt} field contains the format code (for
\texttt{\%s} this field will contain the value \texttt{s}).

	\item The \texttt{pfd\_flags} field contains a set of flags. As with
\texttt{printf} in the C language, formatting flags can be used to control
whether, for example, the printed values are left aligned or preceeded by
zeroes.

	\item The \texttt{pfd\_width} field contains the width, for example
when printing \texttt{printf("\%3d", value);} the \texttt{pfd\_width} field
contains three (3). 

	\item The \texttt{pfd\_dynwidth} field contains the dynamic width;
for example, when printing \texttt{printf("\%*d", width, value);} the dynwidth
is the value of \texttt{width}. 

	\item The \texttt{pfrdv\_prec} field contains the precision.

	\item The \texttt{pfd\_conv} field points to a data structure used to
handle a specifc format code (such as \texttt{\%s}). The \texttt{printf} format
conversion dictionary (\texttt{\_dtrace\_conversions}) can be found in the
file \texttt{dt\_prinf.c}.

	\item The \texttt{pfd\_rec} field contains a pointer to the record that
the format code (modifiers and flags) applies to.

	\item The \texttt{pfd\_next} field contains a pointer to the next
\texttt{struct~\textit{pfargd}} in the list or \texttt{NULL} if there are no
further entries.

\end{itemize}

\section{Aggregations}
\label{sec:trace-aggregations}

\subsection{OpenDTrace aggregation trace buffer}
\label{subsec:agg_trace_buffer}

Figure~\ref{fig:agg_trace_buffer} presents an overview of a OpenDTrace aggregation
trace buffer:

\begin{figure}[!h]
	\centering
	\begin{tikzpicture}
		\matrix [matrix of nodes] {
			\node[
				below,
				rectangle split,
				rectangle split horizontal=true,
				minimum height=1cm,
				align=center,
				rectangle split,
				rectangle split parts=7,
				rectangle split part fill={red!20,blue!20,green!20,none,red!20,blue!20,green!20},
				draw
			] (barr1) 
			{	\nodepart[nn]{one}\texttt{AGGID}
				\nodepart[nn]{two}\texttt{key}
				\nodepart[nn, text width=3cm]{three}\texttt{value}
				\nodepart[nn]{four}\texttt{...}
				\nodepart[nn]{five}\texttt{AGGID}
				\nodepart[nn]{six}\texttt{key}
				\nodepart[nn,text width=3cm]{seven}\texttt{value}
			}; \\
			% Next row
			\node[
				fit=(barr1),
				inner xsep=0pt,
				minimum height=3cm,
				align=center,
				fit={(barr1)},
				draw] {\texttt{...}}; \\
			% Next row
			\node[
				fit=(barr1),
				inner xsep=0pt,
				minimum height=1cm,
				align=center,
				fill=blue!20,
				draw] (free) {\texttt{Aggregation keys}}; \\
			% Next row
			\node[
				fit=(barr1),
				inner xsep=0pt,
				minimum height=1cm,
				align=center,
				fill=red!20,
				draw] {\texttt{Aggregation metadata}}; \\
		};

	\draw [->,transform canvas={xshift=-25pt}] (free.north west) -- ([yshift=25pt] free.north west) node[midway,above=10pt]{\texttt{keys}};

	\draw [->,transform canvas={xshift=-25pt}] (barr1.south west) -- ([yshift=-25pt] barr1.south west) node[midway,below=10pt]{\texttt{records}};

	\draw [decorate,decoration={brace,raise=5pt}] (barr1.north west) -- (barr1.three split north) node[midway,above=10pt]{$Record_0$};

	\draw [decorate,decoration={brace,raise=5pt}] (barr1.four split north) -- node[above=10pt]{$Record_n$} (barr1.north east);
	
	\end{tikzpicture}
	\caption{OpenDTrace aggregation trace buffer.}
	\label{fig:agg_trace_buffer}
\end{figure}

\begin{itemize}

	\item{\texttt{AGGID}}: identifies the aggregation corresponding to the
trace record; the identifier type is \texttt{dtrace\_aggid\_t} which corresponds
to a unsigned 32 bit integer.

	\item{\texttt{key}:} the key of the aggregation entry. (Note that
OpenDTrace supports compound keys such as \texttt{@a[probefunc, arg0] = count();}.)

	\item{\texttt{value}:} the value corresponding to the key. In the case of
an aggregation such as \texttt{count()} or \texttt{min()} the value contains
the current count or minimum value respectively. In other cases, such as
computing the average, the value may consist of a tuple. For example, as
computing an average does not distribute over addition, when computing
\texttt{avg(timestamp - self->ts)} the aggregating function stores both the
running sum of \texttt{timestamp - self->ts} and the number of times the
\texttt{avg()} function was called. These values are then used to compute the
average at the point when the aggregation is actual printed (by \texttt{libdtrace}).

\end{itemize}

Note that neither the length of the key (or keys) or the length of the aggregation 
value is specified by the OpenDTrace trace aggregation buffer.
Instead, a separate stream of metadata is used to interpret the trace buffer.
The data structures describing the metadata stream are described in
Section~\ref{subsec:agg_data_structures}.

\subsection{Data structures}
\label{subsec:agg_data_structures}

\subsubsection{\texttt{struct~\textit{dtrace\_aggdesc}}}
\label{subsubsec:dtagd-aggdesc}

The \texttt{struct~\textit{dtrace\_aggdesc}} (Figure~\ref{fig:dtagd})
contains metadata required to interpret trace records from a given
aggregation.  The \texttt{struct~\textit{dtrace\_aggdesc}} is returned
from the kernel to userspace by invoking the \texttt{AGGDESC} ioctl.

\begin{figure}[!h]
	\centering
	\begin{tikzpicture}[->, -{Latex}]

		\node[split=9,label=north:\texttt{struct~\textit{dtrace\_aggdesc}}] (aggdesc) at (6,0) {
		\textit{dtagd\_name: char *}
		\nodepart{two}\textit{dtagd\_varid: dtrace\_aggvarid\_t}
		\nodepart{three}\textit{dtagd\_flags: int}
		\nodepart{four}\textit{dtagd\_id: dtrace\_aggid\_t}
		\nodepart{five}\textit{dtagd\_epid: dtrace\_epid\_t}
		\nodepart{six}\textit{dtagd\_size: uint32\_t}
		\nodepart{seven}\textit{dtagd\_nrecs: int}
		\nodepart{eight}\textit{dtagd\_pad: uint32\_t}
		\nodepart{nine}\textit{dtagd\_rec[1]: dtrace\_recdesc\_t}};

	\end{tikzpicture}
	\caption{Data structure used to describe an aggregationn.}
	\label{fig:dtagd}
\end{figure}

\begin{itemize}

	\item The \texttt{dtagd\_name} field is used to store the name of the
aggregation (that is, the identifier used in the OpenDTrace script, for example,
\texttt{@a}). It should be noted that this name is not known within the kernel and
is therefore not returned by the \texttt{AGGDESC} ioctl call. Correlating the
\texttt{AGGID} (stored in the \texttt{dtagd\_id} field) with the userspace
identifier and name is described in further detail below.

	\item The \texttt{dtagd\_varid} field contains id assigned to the
aggregation during the compilation. This value is not known within the kernel
and is therefore not returned by the \texttt{AGGDESC ioctl}.

	\item The \texttt{dtagd\_flags} field contains a set of flags that apply to
the aggregation; in the current implementation a single flag 
\texttt{DTRACE\_AGD\_PRINTED} is present. When set,
\texttt{DTRACE\_AGD\_PRINTED} indicates that the aggregation has been printed.

	\item The \texttt{dtagd\_id} field contains the identifier assigned to the
aggregation. Currently OpenDTrace identifies both aggregations and enablings with
simple numerical identifiers (32-bit unsigned integers). The value of these
identifiers depends on the current kernel state. For example aggregation identifiers
are assigned by a kernel unit allocator (\texttt{vmem\_alloc()} on Illumos and
\texttt{alloc\_unr()} of FreeBSD). The value returned by the allocator is
clearly dependent on previous allocated/freed values, which in turn depends on
the current OpenDTrace enablings.

	\item The \texttt{dtagd\_epid} field contains the identifer of the enabling
(that is, the enabled probe identifier).

	\item The \texttt{dtagd\_size} field contains the size of the aggregation;
thta is, the total size of the aggregation trace record (see
Figure~\ref{fig:agg_trace_buffer}).

	\item The \texttt{dtagd\_nrecs} field contains the number of records in
the \texttt{dtagd\_rec} field.

	\item The \texttt{dtagd\_pad} field points to a data structure used to
handle a specifc format code (such as \texttt{\%s}).\footnote{The printf format
conversion dictionary (\texttt{\_dtrace\_conversions}) can be found in the
		file \texttt{dt\_printf.c}}.

	\item The \texttt{dtagd\_rec} field is a variable sized array (of
\texttt{dtagd\_nrecs} entries); this data structure is described in
Section~\ref{subsubsec:dtrd}.  As with probes, the
\texttt{struct~\textit{dtrace\_recdesc}} data structures contain metadata
necessary to parse trace records in the aggregation trace buffer (see
Figure~\ref{fig:agg_trace_buffer}). In the case of aggregations, these data
structures define the location of the key (or sets of keys) and the value; note
that the value is always defined by the last record description.

\end{itemize}

Both the kernel and userspace independently name aggregations. In kernel,
aggregations are named with a kernel unit allocator (such as
\texttt{alloc\_unr} in FreeBSD). In contrast, userspace assigns an identifer to
the aggregation at compilation time. The \texttt{dtagd\_varid} field is used to
contain the compile time identifier for the aggregation. This is determined by
inspecting the \texttt{dtag\_rec[0].dtrd\_uarg} field; that is, the first
record description's \texttt{dtrd\_uarg} field. The value of
\texttt{dtrd\_uarg} is cast as a \texttt{dtrace\_stmdesc\_t} allowing the
compiler assigned identifier and the user assigned name of the aggregation to be
determined.

Whith anonymous enablings the connection between the aggregation identifiers
created at compilation and execution time is broken. And instead all
aggregations are assigned \newline
\texttt{DTRACE\_AGGVARIDNONE}\footnote{Note as the aggregation name can't be
determined it cannot be included in the printed output.}

\subsubsection{\texttt{struct~\textit{dtrace\_aggkey}}}
\label{subsubsec:dtagd-aggkey}

The \texttt{struct~\textit{dtrace\_aggkey}} (Figure~\ref{fig:dtagd}) is used to
represent a key within a given aggregation. This data structure is used solely
within the kernel and thus should be considered as part of the implementation.

\begin{figure}[!h]
	\centering
	\begin{tikzpicture}[->, -{Latex}]

		\node[split=5,label=north:\texttt{struct~\textit{dtrace\_aggkey}}] (aggdesc) at (6,0) {
		\nodepart{one}\textit{dtak\_hashval: uint32\_t} 
		\nodepart{two}\textit{dtak\_action: uint32\_t}
		\nodepart{three}\textit{dtak\_size: uint32\_t}
		\nodepart{four}\textit{dtak\_data: caddr\_t}
		\nodepart{five}\textit{dtak\_next: struct~dtrace\_aggkey~*}};

	\end{tikzpicture}
	\caption{Data structure respresenting a key within the aggregation hash table.}
	\label{fig:dtagd-key}
\end{figure}

\begin{itemize}

	\item The \texttt{dtak\_hashval} field contains the hash value of the key;
the hash value is computed using the Jenkins's ``one-at-a-time'' hash function.

	\item The \texttt{dtak\_action} field identifies the aggregating function
that applies to this aggregation. The \texttt{dtak\_action} fiels is 4-bits in
length allowing 16 aggregation functions, of which 9 are currently defined:

	\begin{enumerate}

		\item \texttt{count()},  
		\item \texttt{min()},
		\item \texttt{max()},
		\item \texttt{avg()},
		\item \texttt{sum()},
		\item \texttt{stddev()},
		\item \texttt{quantize()} power of 2 quantization,
		\item \texttt{lquantize()} linear quantization and
		\item \texttt{llquantize()} log-linear quantization.

	\end{enumerate}

	\item The \texttt{dtak\_size} field contains an offset from the start of
the aggregation record to the value, thus it is the combined size of the
aggregation and the key (or set of keys).

	\item The \texttt{dtak\_data} field contains a pointer to the data
corresponding to this key; that is, the key's corresponding record in trace
buffer (see FIgure~\ref{fig:agg_trace_buffer}).

	\item The \texttt{dtak\_next} field contains a pointer to the next
aggregation key in this list (the hash table is implemented with separate
chaining using a linked list).

\end{itemize}

\subsubsection{\texttt{struct~\textit{dtrace\_aggbuffer}}}
\label{subsubsec:dtagb}

The \texttt{struct \textit{dtrace\_aggbuffer}} (Figure~\ref{fig:dtagd-key})
specifies the metadata for an aggregation. The data structure is used solely
within the kernel and thus should be considered as part of the implementation.
\footnote{The comments within the code suggest that the
userspace copy of the aggregation buffer doesn't contain the hash map and
associated metadata. However, as the \texttt{AGGSNAP} ioctl is practical
identical to the \texttt{BUFSNAP} ioctl, it appears that the data is in the
buffer but unsused by \texttt{libdtrace}}.

\begin{figure}[!h]
	\centering
	\begin{tikzpicture}[->, -{Latex}]

		\node[split=3,label=north:\texttt{struct~\textit{dtrace\_aggbuffer}}] (aggdesc) at (6,0) {
		\textit{dtagb\_hashsize: uintptr\_t}  
		\nodepart{two}\textit{dtagb\_free: uintptr\_t}
		\nodepart{three}\textit{dtagb\_hash: dtrace\_aggkey\_t~**}};

	\end{tikzpicture}
	\caption{Data structure used to describe an aggregationn.}
	\label{fig:dtagd-desc}
\end{figure}

\begin{itemize}

	\item The \texttt{dtagb\_hashsize} field is used to store the number of
buckets in the hashtable used to stored aggregations; in the current
implementation the hash table accounts for approximately $\tfrac{1}{8}$ of the
total buffer size\footnote{Despite the comment suggesting this value may be 
changed as a result of performance analysis, there is no evidence that this
heuristic has ever been evaluated.}.

	\item The \texttt{dtagb\_free} field contains a pointer to the location
in the OpenDTrace aggregation buffer where new aggregation keys are allocated;
as show in Figure~\ref{fig:agg_trace_buffer} allocation of keys occurs from the 
start of the hash table upwards towards the aggregation records.

	\item The \texttt{dtagb\_hash} field contains a pointer to the hash table
used to store aggregations.

\end{itemize}

The relationship between the \texttt{dtagb\_free} and \texttt{dtagb\_hash}
fields and the OpenDTrace aggregation buffer are shown in
Figure~\ref{fig:dtagb_detail}.

\begin{figure}[!h]
	\centering
	\begin{tikzpicture}
		\matrix [matrix of nodes,nodes={align=center,draw}] {
			% Next row
			\node[
				minimum width=11cm,
				fill=green!20] (records) {\texttt{Aggregation records}}; \\
			% Next row
			\node[
				fit=(records),
				inner xsep=0pt,
				minimum height=3cm] {\texttt{...}}; \\
			% Next row
			\node[
				fit=(records),
				inner xsep=0pt,
				fill=blue!20] (free) {\texttt{Aggregation keys}}; \\
			% Next row
			\node[
				fit=(records),
				inner xsep=0pt,
				fill=red!20,
				minimum height=0.5cm] (hashtable1) {\texttt{struct~\textit{dtrace\_aggkey}~*}}; \\
			% Next row
			\node[
				fit=(records),
				inner xsep=0pt,
				fill=red!20,
				minimum height=0.5cm] {\texttt{...}}; \\
			% Next row
			\node[
				fit=(records),
				inner xsep=0pt,
				fill=red!20,
				minimum height=0.5cm] (hashtablen) {\texttt{struct~\textit{dtrace\_aggkey}~*}}; \\
			% Next row
			\node[
				fit=(records),
				inner xsep=0pt,
				fill=red!20,
				minimum height=0.5cm] (dtagb) {\texttt{struct~\textit{dtrace\_aggbuffer}}}; \\
		};

	\draw [->,transform canvas={xshift=-5pt}] ([xshift=-25pt] free.north west) -- (free.north west) node[midway,left=15pt]{\texttt{dtagb\_free}};

	\draw [->,transform canvas={xshift=-5pt}] ([xshift=-25pt] hashtable1.north west) -- (hashtable1.north west) node[midway,left=15pt]{\texttt{dtagb\_hash}};

	\draw [db] (hashtable1.north west) -- (hashtablen.south west) node[midway,left=10pt]{\texttt{dtagb\_hashsize}};

	\end{tikzpicture}
	\caption{OpenDTrace aggregation trace buffer.}
	\label{fig:dtagb_detail}
\end{figure}
