\clearpage
\phantomsection
\addcontentsline{toc}{subsection}{STGS}
\label{insn:stgs}
\subsection*{STGS: store a value into a variable}

\subsubsection*{Format}

\textrm{STGS \%rd, \%r1, \%r2}

\begin{center}
\begin{bytefield}[endianness=big,bitformatting=\scriptsize]{32}
\bitheader{0,7,8,23,24,31} \\
\bitbox{8}{0x2A}
\bitbox{16}{var}
\bitbox{8}{rd}
\end{bytefield}
\end{center}

\subsubsection*{Description}
The \instruction{stgs} instruct stores a value from the
\registerop{rd} register into a user defined variable.

\subsubsection*{Pseudocode}

\begin{verbatim}
var = %rd
\end{verbatim}

\subsubsection*{Constraints}

\subsubsection*{Failure modes}

This instruction has no run-time failure modes beyond its constraints.
