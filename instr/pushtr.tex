\clearpage
\phantomsection
\addcontentsline{toc}{subsection}{PUSHTR}
\label{insn:pushtr}
\subsection*{PUSHTR: push a reference onto the stack}

\subsubsection*{Format}

\textrm{PUSHTR \%rd, \%r1, \%r2}

\begin{center}
\begin{bytefield}[endianness=big,bitformatting=\scriptsize]{32}
\bitheader{0,7,8,15,16,23,24,31} \\
\bitbox{8}{0x30}
\bitbox{8}{r1}
\bitbox{8}{r2}
\bitbox{8}{rd}
\end{bytefield}
\end{center}

\subsubsection*{Description}

The \instruction{pushtr} instruction pushes a reference, contained in

the \registerop{rd} register onto the stack.
\subsubsection*{Pseudocode}

\begin{verbatim}
stack[++index] = %rd
\end{verbatim}

\subsubsection*{Constraints}

\subsubsection*{Failure modes}

This instruction has no run-time failure modes beyond its constraints.
