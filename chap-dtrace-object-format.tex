\section{Introduction}
\label{sec:dof-intro}

DTrace programs can be persistently encoded in the DOF format so that
they may be embedded in other programs (for example, in an ELF file)
or in the dtrace driver configuration file for use in anonymous
tracing.  The DOF format is versioned and extensible so that it can be
revised and so that internal data structures can be modified or
extended compatibly.  All DOF structures use fixed-size types, so the
32-bit and 64-bit representations are identical and consumers can use
either data model transparently.

\subsection{Stable Storage Foramt}
\label{sec:dof-stable-storage}

\begin{figure}[h]
  \centering
  \includegraphics[width=.8\textwidth]{dof-stable-format}
  \caption{Stable Storage Format}
  \label{fig:stable-storage-format}
\end{figure}


When a DOF file resides on stable storage it is stored in the format
shown in ~\ref{fig:stable-storage-format}. The file header stores
meta-data including a magic number, data model for the
instrumentation, data encoding, and properties of the DIF code within.
The header describes its own size and the size of the section headers.
By convention, an array of section headers follows the file header,
and then the data for all loadable sections and unloadable sections.
This permits consumer code to easily download the headers and all
loadable data into the DTrace driver in one contiguous chunk, omitting
other extraneous sections.

The section headers describe the size, offset, alignment, and section
type for each section.  Sections are described using a set of \verb|#defines|
that tell the consumer what kind of data is expected.  Sections can
contain links to other sections by storing a \verb|{dof_secidx_t|, an index
into the section header array, inside of the section data structures.
The section header includes an entry size so that sections with data
arrays can grow their structures.

The DOF data itself can contain many snippets of DIF (i.e. >1 DIFOs),
which are represented themselves as a collection of related DOF
sections.  This permits us to change the set of sections associated
with a DIFO over time, and also permits us to encode DIFOs that
contain different sets of sections.  When a DOF section wants to refer
to a DIFO, it stores the \verb|dof_secidx_t| of a section of type
\verb|DOF_SECT_DIFOHDR|.  This section's data is then an array of
\verb|dof_secidx_t|'s which in turn denote the sections associated
with this DIFO.

This loose coupling of the file structure (header and sections) to the
structure of the DTrace program itself (ECB descriptions, action
descriptions, and DIFOs) permits activities such as relocation
processing to occur in a single pass without having to understand D
program structure.

Finally, strings are always stored in ELF-style string tables along
with a string table section index and string table offset.  Therefore
strings in DOF are always arbitrary-length and not bound to the
current implementation.


%%% Local Variables:
%%% mode: latex
%%% TeX-master: "dtrace-specification"
%%% End:
